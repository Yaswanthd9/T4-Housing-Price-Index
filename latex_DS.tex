
\documentclass[12pt]{article}
\newcommand\tab[1][.50cm]{\hspace*{#1}}
\usepackage{amsmath}
\usepackage{amsthm}
\usepackage[T1]{fontenc}
\usepackage{times}
\usepackage{pgfplots}
\pgfplotsset{compat = newest}
\usetikzlibrary{positioning, arrows.meta}
\usepgfplotslibrary{fillbetween}
\usepackage{lmodern}
\usepackage{ragged2e}
\usepackage{multicol, caption, booktabs}
\usepackage[flushleft]{threeparttable}
\usepackage[utf8]{inputenc}
\usepackage{siunitx}
\usepackage{setspace}
\usepackage[export]{adjustbox}
%\usepackage[usenames,dvipsnames]{xcolor}
\usepackage[affil-it]{authblk} 
\setlength{\columnsep}{1cm}
\newcommand{\dd}[1]{\mathrm{d}#1}
\usepackage[portrait, margin=1in]{geometry}
\graphicspath{ {C:/Users/Stu Lucko/Documents/GitHub/22SP_6103_11T_iDM/} }
\graphicspath{ {C:/Users/Stu Lucko/Documents/GitHub/Econometrics/} }
\newenvironment{Figure}
{\par\medskip\noindent\minipage{\linewidth}}
{\endminipage\par\medskip}
\usepackage{amssymb}
\usetikzlibrary{calc} 
%\onehalfspacing%
\begin{document}


%$$ d=2r\ arcsin \left(\sqrt{sin^{2} \left(\frac{\phi_{2}-\phi_{1}}25}\right)} \right)$$

%$$\phi$$

$$ d=2r\ arcsin \left(\sqrt{sin^{2} \left( \frac{\phi_{2}-\phi_{1}} {2} \right) +cos(\phi_1)cos(\phi_2)sin^2 \left( \frac{\lambda_{2}-\lambda_{1}} {2} \right)       } \right)$$\\\\
\centering
\footnotesize
where $\phi_1$ and $\phi_2$ correspond to latitude 1 and latitude 2,\\ 
 $\lambda_1$ and $\lambda_2$ corresond to longitude 1 and longitude 2, \\ 
and $r$ is the radius of the Earth. \\ 
\bigskip
The latitude and longitude coordinates for each house and each metro station\\
are substituded into the haversine fuction for $(\phi_1,\ \lambda_1) \ \& \  (\phi_2,\ \lambda_2)$ \\
respectively. The distance from a single house to all 91 metro stations\\ are 
calculated, then the minumum distance is selected and added to the dataframe.














\end{document}